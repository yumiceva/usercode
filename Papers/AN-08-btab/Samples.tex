\section{Selection Criteria and Simulation Datasets}
\label{sec:samples}

This study is based on three different Monte Carlo samples denoted QCD, 
inclusive $t\bar{t}+0$jets and $pp\rightarrow \mu +X$. The QCD sample 
was generated in different $\hat{p}_T $ bins at generator level. All
samples were generated with PYTHIA~\cite{ref:pythia}, except for the $t\bar{t} $ sample 
that was generated with ALPGEN~\cite{ref:alpgen}. All samples were simulated with 
CMSSW\_1\_4\_X and reconstructed with CMSSW\_1\_5\_2 as part of the CSA07
production. The $pp\rightarrow \mu +X $ sample is QCD minbias events with
muon $p_T > 3$~GeV/c. The muon is selected at the generator level. This
sample does not include muons produced by pion or kaon decays or by
punch-through. Thus, the light flavor jet contribution is much
lower than in the inclusive QCD sample.

The analysis is based on samples that have at least two reconstructed jets 
and a non-isolated muon close to one of the jets. The jets are reconstructed 
using the ITERATIVECONE5 algorithm~\cite{ref:iterativecone5} and corrected by a generator to 
reconstruction calibration factor. The jets and the muons in the event are
selected using the criteria described in the CMS Analysis 
Note~\cite{ref:btag_oldnote}.

Two samples are used in the analysis, defined as follows:
\begin{itemize} 
\item The  muon-in-jet+away-jet sample contains two reconstructed jets
and a non-isolated muon with $\Delta R(\mu,{\rm jet})<0.4$. In case 
more than one muon is found within a cone of $\Delta R<0.4$ of the jet, the 
muon with the highest $p_T$ is taken. For this study, the jet containing the 
muon will be denoted ``muon-jet",  and the other jet will be denoted the 
``away-jet''. If in a given event both jets contain muons, both will be 
counted as muon-jet. 
\item The muon-in-jet+tagged-away-jet sample is a subset of the 
muon-in-jet+away-jet sample, where the away-jet is tagged as a $b$ jet. 
\end{itemize}

Table~\ref{tab:samples} summarizes the samples used, and the number 
of events available.
For muon-jets the \ptrel is defined as the transverse momentum of the muon 
relative to the direction of the total muon-jet momentum vector,
 
\begin{equation}
p_{Trel}=\frac{|\vec{p^{\mu}} \times \vec{p^{\mu + {\rm jet}}}|}{|p^{\mu + {\rm jet}}|}.
\end{equation}

\begin{table}[bth]
 \begin{center}
 \begin{tabular}{l|r|r|r}
Sample                 & Events  & muon-in-jet & muon-in-jet+away-jet \\ \hline
QCD $\hat{p_T}$ 15-20  & 1286976 &    398      &       15 \\
QCD $\hat{p_T}$ 20-30  & 750966  &   864       &      59 \\
QCD $\hat{p_T}$ 30-50  & 1160479 &   5334      &      649 \\
QCD $\hat{p_T}$ 50-80  & 900240  & 11428       &    2390 \\
QCD $\hat{p_T}$ 80-120  & 1248757 &  27858     &      8592 \\
QCD $\hat{p_T}$ 120-170  & 1260951&   41667    &      16946 \\
QCD $\hat{p_T}$ 170-230  & 837547 &  36071     &     17714 \\
QCD $\hat{p_T}$ 230-300  & 760840 &  40190     &     22891 \\
QCD $\hat{p_T}$ 300-380  & 1225037&   73804    &      46954 \\
QCD $\hat{p_T}$ 380-470  & 1196202&   80258    &      55228 \\
QCD $\hat{p_T}$ 470-600  & 1226113&   90975    &      66464 \\
QCD $\hat{p_T}$ 600-800  & 546080 &  44122     &     33859 \\
QCD $\hat{p_T}$ 800-1000  & 717958  &   63664  &        50699 \\ \hline
inclusive $t\bar{t}+0$jets& 1511164 &    339947 &       259662 \\ \hline
$pp\rightarrow \mu +X$    & 18664210 &    611771 &      71519 \\ \hline

 \end{tabular}
 \end{center}
\caption[]{Summary of the total number of events from the CSA07 MC samples.}
\label{tab:samples}
\end{table}

